\subsection{Eval/Apply}

\paragraph{Principe général}

Pour calculer l'évaluation d'une expression \texttt{Lisp}, nous appellons
récursivement la fonction \emph{eval\_env} sur l'objet correspondant.
La fonction \emph{eval\_env} prend en paramètre l'environnement
courant, et un objet à évaluer. Cet objet peut représenter plusieurs
choses: un atome ou une liste.

\paragraph{Évaluation des atomes}

Si cet objet est un atome, on distingue deux sous-cas: il peut être un
atome constant (comme une chaîne de caractères, un nombre ... ) , auquel
cas nous renvoyons sa valeur, ou une variable. Nous retournons
alors l'évaluation de l'objet qui lui est lié dans l'environnement
courant.

\paragraph{Évaluation des listes}

Si cet objet est une liste, nous reconnaissons son premier élément. Il
peut représenter une subroutine, un mot-clé du langage, ou encore une
variable ou une fonction. Pour reconnaître cet objet, nous utilisons
des \emph{map} associant à une chaîne de caractères la fonction à appliquer à
la queue de la liste. L'utilisation d'une \emph{map} présente plusieurs
avantages. Premièrement, un léger gain de rapidité dans l'éventualité
d'un passage à une plus grande échelle, la recherche dans une \emph{map} se
faisant de manière dichotomique et non linéaire. De plus, cela permet
d'allèger et de clarifier considérablement la fonction
\emph{eval\_env} en factorisant le code. 

Si la tête de la liste ne correspond à aucun des symboles existants
déja dans le langage, c'est alors une variable ou une fonction. En
évaluant cette expression, nous obtenons une fonction, que nous
pouvons ensuite appliquer à la queue de la liste.

\paragraph{Application d'une fonction}

Pour appliquer une fonction à une liste d'arguments, nous commençons
par évaluer chacun des arguments afin d'obtenir la liste de valeurs
correspondante. Cette évaluation multiple se fait à l'aide de la
fonction \emph{map\_eval}. 

Une fois la liste des valeurs récupérée, nous appelons
\emph{apply}. Cette fonction prend en paramètres une fonction
\texttt{(lambda (parameters) (body))} à appliquer, la liste des
valeurs \texttt{(values)} des paramètres précédemment calculée et
l'environnement courant. Elle crée ensuite un nouvel environnement en
appelant la méthode \emph{set\_global}, et lie dans le nouvel
environnement chacune des variables de \texttt{(parameters)} avec sa
valeur donnée dans \texttt{(values)}.

Ces liaisons sont réalisées par la fonction \emph{zip}. Il suffit
ensuite de renvoyer l'évaluation \texttt{(body)} dans l'environnement
étendu, en ayant au préalable remis l'environnement dans l'état
initial grâce à la méthode \emph{collapse}

\paragraph{Variables locales}

Il est possible en \texttt{Lisp} de créer des variables locales en
utilisant le mot-clé \texttt{let}. Ainsi, pour évaluer l'objet
\texttt{(let (local\_variables) (body))} (où
\texttt{(local\_variables)} est une liste de couples \texttt{(variable
  expression)}), nous commençons par créer un nouvel environnement.
Dans cet environnement, nous lions \texttt{variable} avec l'objet
résultant de l'évaluation de \texttt{expression}. Nous renvoyons
ensuite l'évaluation de \texttt{(body)} dans l'environnement étendu
après avoir supprimé ledit environnement. 
