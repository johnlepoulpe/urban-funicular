Pour effectuer un comparaison sur le nombre de lignes de codes des
programmes C et C++, il nous faut faire un test de Mann-Whitney (grâce
à la fonction \emph{wilcox.test} en R. Pour que ce test soit valide,
nous sommes obligés de supposer que les deux échantillons considérés
suivent une loi normale, de variance identique. De plus, les mesures
sont bien idépendantes les unes des autres.  

Nous faisons pour hypothèse nulle ``Il n'y a pas de différences
significatives du nombre de lignes de code dupliquées en fonction du
langage de programmation utilisé.''. La p-value retournée par le test
est de 0.94, au risque d'erreur 5\%, nous ne pouvons donc pas rejeter
l'hypothèse nulle. Nous en concluons que le nombre de lignes de codes
duppliquées et le langage de programmation sont décorrélés.

Nous noterons que la longueur des codes biaise l'interprétation de la
mesure: en effet, plus un code est long, plus le nombre de lignes
duppliquées a de chance d'être élevé, aussi, un grand nombre de lignes
de codes duppliquées ne reflète pas une mauvaise technique de
programmation, tout dépend de la taille du code!  Nous avons donc
aussi comparé le pourcentage des lignes de codes écrits en C et en
C++.  

L'hypothèse nulle devient ``Il n'y a pas de différences significatives
du pourcentage de lignes de code dupliquées en fonction du langage de
programmation utilisé.''.  La p-value retournée par le test est de
0.41, au risque d'erreur 5\%, nous ne pouvons donc pas rejeter
l'hypothèse nulle. Nous en concluons que le langage de programmation et
le pourcentage de lignes de code dupliquées sont décorréllés.

