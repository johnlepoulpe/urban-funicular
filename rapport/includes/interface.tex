\section{Interface}
\label{sec:description}

Dans cette section nous discuterons des différentes interfaces entre
l'utilisateur et l'interprète.  Nous commencerons par décrire notre
utilisation de l'analyseur lexical \emph{Flex} et l'analyseur
syntaxique \emph{Bison}. Puis nous continuerons sur la description du
\emph{toplevel} qui permet d'initialiser les objets et les paramètres
utiles au fonctionnement de l'interpréteur mais gère également la
boucle d'interprétation.

\subsection{Parsing et Lexing}

\paragraph{Nil et t}
Au sein du lexer nous avons décider d'ajouter la liste vide \emph{()} au token
définissant NIL car nous avions lors de l'affichage deux façons de retourner
\emph{()}: soit en imprimant \emph{nil}, soit en imprimant le symbole
\emph{()}.

Nous avons également ajouté un token pour \emph{t}
et la règle permettant au parser de reconnaître ce
token et d'appliquer l'objet \emph{t()} au résultat.

\paragraph{Macro expansion}
Un des objectifs du projet était de gérer la macro-expansion du
caractère \emph{'}.  Pour cela nous avons rajouté une règle à
\emph{Bison} pour transformer une expression du type \emph{'exp} en
\emph{(quote(exp))}.

\subsection{Toplevel}

\paragraph{Appel à l'interpréteur}
Nous avons fait le choix d'encapsuler l'appel à l'interprète et la
boucle d'interprétation dans la fonction \emph{run\_toplevel}.

\paragraph{Initialisation et aboutissement de l'interpréteur}
Lors de la mise en place du \emph{toplevel}, nous initialisons les structures nécéssaires à
l'interprète, comme l'environnement, le ramasse-miette et les différents symboles reconnaissables en tant
qu'instructions \texttt{Lisp} et leurs traductions en tant que
fonctions \emph{C++}.

De plus, lors de l'arrêt de l'interprétation, nous effectuons une
opération de nettoyage à la fois au niveau des \emph{Cell}, et des maillons de la liste chainée de l'environnement.\\

\paragraph{Gestion des exceptions}
Nous avons décidé de gérer la quasi-totalité du rattrapage des erreurs
dans le \emph{toplevel}, qui, étant situé juste en-dessous du main, nous centralisons la gestion
d'erreurs et l'affichage des messages destinés à l'utilisateur.

\paragraph{Gestion de la mémoire}
La gestion du ramasse-miette et de la mémoire se fait également au niveau du
\emph{toplevel} car on peut y déterminer les états stables de la mémoire (avant
la lecture d'une expression ou après son évaluation). Ainsi, dans notre boucle
d'interprétation, nous avons décider d'appeler une directive de nettoyage du
ramasse-miette avant chaque lecture d'expression.

\paragraph{Gestion des directives haut niveau}
Dans l'interpréteur \texttt{Lisp}, nous gérons les directives \texttt{setq} et
\texttt{defun} ou \texttt{exit} (qui permet de quitter le système), de manière
différente, car elles ne renvoient pas d'objet. Nous interdisons donc à
l'utilisateur de définir ces directives à l'intérieur d'autres expressions,
elles sontreconnues et traitées grâce à une structure de \emph{map} associant le
symbole de la directive à sa fonction.
