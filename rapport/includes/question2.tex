Afin de calculer le pourcentage de ligne dupliquées, nous avons créé
une nouvelle table, contenant les noms des programmes, leurs
dommaines, les langages dans lesquels ils ont été écrits et leurs
nombres de lignes dupliquées ainsi que leurs nombres de lignes
totales. Nous avons enlevé de cette nouvelle table de données les
programmes pour ayant un champs non renseigné (à l'aide de la commande
\emph{na.omit}). Nous avons ensuite rajouté une colonne à celle ci
contenant les pourcentages de lignes de codes dupliquées, avant
de créer un histogrammes de ces pourcentages pour chaque programme,
ordonné par ordre décroissant (figure \ref{pdlin_prog}).
