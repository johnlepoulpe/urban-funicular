Nous vouons explorer l'hypothèse selon laquelle les codes longs
comporteraient plus de warnings à la compilation que les courts. Nous
avons commencé a séparer notre table en deux, l'une pour les
programmes au nombre de lignes de codes inférieur à la médiane et
l'autre pour les autres programmes. 

Nous avons ensuite effectué trois test de Mann Whitney, un pour chaque
type de warning (Clang, MinorWarning de gcc, MajorWarning de gcc),
avec pour hypothèse nulle ``Il n'y a pas de différence significatives
du nombre de warning à la compilation entre les programmes de
longueurs supérieure et inférieure à la médiane.''. Nous avons pour
cela encore une fois supposé que les échantillons considérés suivaient
une loi normale de variance identique. Lors du test, nous avons passé
\emph{alternative = "less"} en argument à la fonction
\emph{wilcox.test}, cela correspond au fait que nous nous attendons à
ce que le nombre de warning soit supérieur pour pour les programmes
long.

Pour les Warning Clang, la p-value retournée par le test est de
0.0006216, au risque d'erreur 5\%, nous pouvons donc rejeter
l'hypothèse nulle. Nous en concluons que le nombre de warning Clang
est supérieur pour les programmes longs.

Pour les Warning gcc, la p-value retournée par le test est de 0.6039
et 0.05896 pour les warning mineurs et majeurs respéctivement. Au
risque d'erreur 5\%, nous ne pouvons donc pas rejeter l'hypothèse nulle. Nous
en concluons que le nombre de warning gcc n'est pas sensiblement supérieur pour les
programmes longs.

